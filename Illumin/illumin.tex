%\documentclass[12pt, draft, onecolumn]{IEEEtran}
\documentclass[]{IEEEtran}

\usepackage{cite}

\title{SR 71/72}
\author{Nakul Joshi}
\begin{document}
\maketitle
\begin{abstract}

	Beautiful and fast, the Lockheed Martin SR-71 ``Blackbird'' aircraft was amongst the most successful reconnaissance aircraft ever built. A marvel of engineering, it fulfilled its military role perfectly: it flew too high to be detected by radar and could not be shot down simply because it was faster than any missile in the sky. Retired for political reasons in 1998, it continues to hold the record for fastest manned air-breathing craft. Recently, Lockheed Martin announced the development of the Blackbird's successor, the unmanned SR-72 ``Aurora". This paper examines the various innovations that made possible the unique capabilities of the Blackbird, and takes a glimpse at the possibilities offered by the Mach 6+ Aurora.
	 
\end{abstract}
\begin{IEEEkeywords}
lockheed martin, skunk works, sr, blackbird, aurora
\end{IEEEkeywords}


\section{Introduction}
	\IEEEPARstart{T}{he}	need for a high-speed stealth recon aircraft was recognised by the CIA in 1960, after the downing of a U2 aircraft over Soviet airspace. The U2 was thought to be safe from enemy radar at its 20km altitude, but the relatively slow craft was intercepted by Soviet planes near the Ural mountains~\cite{u2}, causing an escalation of tensions during the Cold War.
	
	

	
\section{Engineering}

\section{Conclusion}


\bibliography{citations}{}
\bibliographystyle{plain}
\end{document}