\documentclass[12pt,letterpaper]{article}
\usepackage{ifpdf}
\usepackage{mla}


\begin{document}

\begin{mla}{Nakul}{Joshi}{Schroeder}{WRIT 340}{\today}{Title Goes Here}

Humanity's demand for energy has grown at a dramatic rate since the industrial revolution. A majority of this energy is obtained from hydrocarbons, which have become very difficult and expensive to obtain as oil companies look further and deeper for sources to replace the ones that have been depleted. In recent years, the controversial process of induced hydraulic fracturing (commonly called `fracking') has emerged as a means of exploiting the previously inaccessible supplies of natural gas embedded in subterranian shale reservoirs. Proponents of the technique have pointed to the possibilities of largely reducing the reliance on imports of natural gas, while also stimulating local economies. However, the risks posed to the environments surrounding fracking sites have not been fully addressed by the industry, and short of an outright ban, heavy regulation of the process is necessary until long-term studies of its effects have been carried out.
\end{mla}
\end{document}