\documentclass[12pt,letterpaper]{article}
\usepackage{ifpdf}
\usepackage{mla}


\begin{document}

\begin{mla}{Nakul}{Joshi}{Schroeder}{WRIT 340}{\today}{Title Goes Here}

Humanity's demand for energy has grown at a dramatic rate since the industrial revolution. A majority of this energy is obtained from hydrocarbons, which have become very difficult and expensive to obtain as oil companies look further and deeper for sources to replace the ones that have been depleted. In recent years, the controversial process of induced hydraulic fracturing (commonly called `fracking') has emerged as a means of exploiting the previously inaccessible supplies of natural gas embedded in subterranian shale reservoirs. Proponents of the technique have pointed to the possibilities of largely reducing the reliance on imports of natural gas, while also stimulating local economies. However, the risks posed to the environments surrounding fracking sites have not been fully addressed by the industry, and short of an outright ban, heavy regulation of the process is necessary until long-term studies of its effects have been carried out.

The fracking process involves creating a solution of water, sand, and various chemicals; the resultant fracking fluid is then injected down wellbores at high pressure. This creates millimetre-wide fissures along shale beds deep underground, allowing the embedded natural gas to migrate into the well, from where it can be extracted relatively easily. Without this enhancement of well output, extracting shale gas would be economically infeasible compared to going after the more limited reserves of conventional gas or importing

\end{mla}
\end{document}