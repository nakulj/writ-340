\documentclass[12pt,letterpaper]{article}

\usepackage{../mla13}
\sources{sourcelist.bib}

\firstname{Nakul}
\lastname{Joshi}
\professor{Schroeder}
\class{WRIT 340}
\title{Title Goes Here}

\begin{document}

\makeheader

Humanity's demand for energy has grown at a dramatic rate since the industrial revolution. A majority of this energy is obtained from hydrocarbons, which have become very difficult and expensive to obtain as oil companies look further and deeper for sources to replace the ones that have been depleted. In recent years, the controversial process of induced hydraulic fracturing (commonly called ``fracking") has emerged as a means of exploiting the previously inaccessible supplies of natural gas embedded in subterranian shale reservoirs. Proponents of the technique have pointed to the possibilities of largely reducing the reliance on imports of natural gas, while also stimulating local economies. However, these transient benefits are not worth the accompanying ecological risks, and continued use of the technology runs counter to the long-term goal of reducing dependance on polluting and non-sustainable energy sources.

Fracking technology developed in the 1940s as a response to the former economic infeasibility of exploiting the abundant domestic reserves of shale gas.
The process begins by creating a solution of water, sand, and various chemicals; the resultant fracking fluid is then injected down wellbores at high pressure. This creates millimetre-wide fissures along shale beds deep underground, allowing the embedded natural gas to migrate into the well, highly enhancing gas production. According to the National Petroleum Council, fracking has the potential to provide for 70\%  \cite{npc}  of the natural gas consumption in the North Amercican continent, and banning it would cut production by half in around five years.
%Further, the spillover effects from investments in the technology have had a positive impact on per capita income \cite{mining}.

However, the petrochemical engineers who develop fracking techniques have a responsibility to look beyond the profit margins of their employers, and ensure that their designs pose minimal harm in the big picture, and in the long run.
As the National Society of Professional Engineers (NSPE) Code of Ethics for Engineers states, 
``Engineering has a direct and vital impact on the quality of life for all people. Accordingly, the services provided by engineers [\ldots] must be dedicated to the protection of the public health, safety, and welfare" \cite{ethics}.
Ethically, then, engineers are obligated to stop the trend of pursuing ever riskier means of depleting the Earth's limited resources, but they have failed in this regard.

The primary concern is contamination of freshwater supplies: besides being a heavily water-intensive process, fracking also places local supplies of water at risk. Oil lobbies have won exemptions from the Clean Water act and the Safe Drinking Water Act \cite[272]{nature}, buying themselves opacity about the safety of the process.
While the precise composition of the fracking fluids used is kept proprietary and thus hidden from public scrutiny, it is known that they are very dangerous: Howarth and Ingraffea state that they contain ``acids, biocides, scale inhibitors, friction reducers and surfactants, [many of which are] toxic, carcinogenic or mutagenic ". They go on to say that the process ``extracts natural salts, heavy metals, hydrocarbons and radioactive materials from the shale, posing risks to ecosystems and public health when these return to the surface"\cite[272]{nature}.
Oil companies maintain that this fluid can be safely contained within the wells: a British Petroleum issue brief proudly claims that
\begin{quote}
[Underground water sources cannot be contaminated] if the well has been properly engineered and constructed. BP wells and facilities are designed, constructed, operated and decommissioned to mitigate the risk that natural gas and hydraulic fracturing fluids enter underground aquifers, including drinking water sources. \cite[4]{bp}
\end{quote}
 However, several cases have been reported \cite[272]{nature} of the fluid leaching into local water bodies. This contamination is a direct threat to residents, with potential to cause respiratory illnesses, birth defects and neurological damage. The stability of local ecosystems is also undermined by such contamination. 

Even out of the proximity of drinking water, fracking carries the risk of causing dangerous seismological activity. Because the fluid is pumped in at such a high rate and at such high pressure, it has the potential to compromise the structural integrity of the rock bed. Combined with any existing fault lines in the area, this could potentially lead to earthquakes and landslides \cite{quake}.
In Ohio, fracking has been correlated with increased seismic activity; the city of Youngstown, which previously had had no recorded earthquakes since 1776, experienced nine noticable quakes in the year 2011, not long after the construction of the Northstar 1 well nearby \cite{quake2}. In reaction, the director of the Ohio Department of Natural Resources made the statement that
``The seismic events are not a direct result of fracking". However, a study conducted later did in fact conclude that
\begin{quote}
The recent earthquakes in Youngstown, Ohio were induced by the fluid injection at a deep injection well [\ldots] the expanding high fluid pressure front increased the pore pressure along its path and progressively triggered the earthquakes. \cite[1]{quake3}
\end{quote}
Even minor quakes could compromise the fluid containment systems, exacerbating the risk to local water supplies. Worst of all, these quakes may be triggered long after the industrial activity itself has ceased, making it extremely difficult to determine a fracking site's future impact on an area.

Oil companies have tried obscuring these concerns with a veil of legislative loopholes and industry-funded research. In 2005, the so-called ``Halliburton Loophole" was passed, denying the Environmental Protection Agency the power to regulate the fracking process. The loophole was part of the Energy Bill passed by then-Vice President Dick Cheney, who happened to have been a former executive at Halliburton, the oil company that invented the fracking process. Such conflicts of interest are abundant-- soon after a UT Austin study led by geologist Charles Groat widely publicized its finding that fracking had not contaminated groundwater, it surfaced that Groat had failed to disclose his position on the board of a company that ran fracking operations \cite{groat}.
By rooting themselves into the legislature as well as academia, two spheres responsible for protecting the public interest, oil companies have demonstrated a ruthlessness in pursuing their own.
In doing so, they have shown that they cannot be trusted with mere regulation; the process has to be banned outright for any real change.




\makeworkscited

\end{document}